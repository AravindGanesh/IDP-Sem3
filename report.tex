\documentclass[twoside,twocolumn]{article}

\usepackage{blindtext} % Package to generate dummy text throughout this template 
\usepackage{flafter} 
\usepackage{graphicx}
\usepackage[sc]{mathpazo} % Use the Palatino font
\usepackage[T1]{fontenc} % Use 8-bit encoding that has 256 glyphs
\linespread{1.05} % Line spacing - Palatino needs more space between lines
\usepackage{microtype} % Slightly tweak font spacing for aesthetics
\input{arduinoLanguage.tex}

\usepackage[english]{babel} % Language hyphenation and typographical rules

\usepackage[hmarginratio=1:1,top=32mm,columnsep=20pt]{geometry} % Document margins
\usepackage[hang, small,labelfont=bf,up,textfont=it,up]{caption} % Custom captions under/above floats in tables or figures
\usepackage{booktabs} % Horizontal rules in tables

\usepackage{lettrine} % The lettrine is the first enlarged letter at the beginning of the text

\usepackage{enumitem} % Customized lists
\setlist[itemize]{noitemsep} % Make itemize lists more compact

\usepackage{abstract} % Allows abstract customization
\renewcommand{\abstractnamefont}{\normalfont\bfseries} % Set the "Abstract" text to bold
\renewcommand{\abstracttextfont}{\normalfont\small\itshape} % Set the abstract itself to small italic text

\usepackage{titlesec} % Allows customization of titles
\renewcommand\thesection{\Roman{section}} % Roman numerals for the sections
\renewcommand\thesubsection{\roman{subsection}} % roman numerals for subsections
\titleformat{\section}[block]{\large\scshape\centering}{\thesection.}{1em}{} % Change the look of the section titles
\titleformat{\subsection}[block]{\large}{\thesubsection.}{1em}{} % Change the look of the section titles

\usepackage{fancyhdr} % Headers and footers
\pagestyle{fancy} % All pages have headers and footers
\fancyhead{} % Blank out the default header
\fancyfoot{} % Blank out the default footer
\fancyhead[C]{4-Bit Ripple Binary Counter $\bullet$ 13 October 2017} % Custom header text
\fancyfoot[RO,LE]{\thepage} % Custom footer text

\usepackage{titling} % Customizing the title section

\usepackage{hyperref} % For hyperlinks in the PDF

\setlength{\droptitle}{-4\baselineskip} % Move the title up

\pretitle{\begin{center}\Huge\bfseries} % Article title formatting
\posttitle{\end{center}} % Article title closing formatting
\title{Electromagnetic Railgun} % Article title
\author{%
\textsc{Aravind Ganesh}\\[1ex] % Your name
%\normalsize IITH \\ % Your institution
\normalsize \href{mailto:ee16tech11026@iith.ac.in}{ee16tech11026@iith.ac.in}
 % Your email address
\and % Uncomment if 2 authors are required, duplicate these 4 lines if more
\textsc{\textbf{Siva Kumar}}\\
 Project advisor\\
\normalsize \href{mailto:ee16btech11006@iith.ac.in}{skumar@iith.ac.in} % Second author's email address
\and % Uncomment if 2 authors are required, duplicate these 4 lines if more
\textsc{Adithya Hosapate}\\[1ex] % Second author's name
%\normalsize IITH \\ % Second author's institution
\normalsize \href{mailto:ee16btech11040@iith.ac.in}{ee16btech11040@iith.ac.in} % Second author's email address
\and % Uncomment if 2 authors are required, duplicate these 4 lines if more
\textsc{Anand N Warrier}\\[1ex] % Second author's name
%\normalsize IITH \\ % Second author's institution
\normalsize \href{mailto:ee16btech11042@iith.ac.in}{ee16btech11042@iith.ac.in} % Second author's email address
\and
\textsc{Deep Diwani}\\[1ex] % Your name
%\normalsize IITH \\ % Your institution
\normalsize \href{mailto:ee16tech11006@iith.ac.in}{ee16tech11006@iith.ac.in}
}

\date{\today} % Leave empty to omit a date
\renewcommand{\maketitlehookd}{%

}

%----------------------------------------------------------------------------------------

\begin{document}

% Print the title
\maketitle

%----------------------------------------------------------------------------------------
%	ARTICLE CONTENTS
%----------------------------------------------------------------------------------------

\section{Introduction}

	A railgun is a device that uses electromagnetic force to launch high velocity projectiles, by means of a sliding armature that is accelerated along a pair of conductive rails. Railguns rely on electromagnetic force to propel a projectile at very high velocities(more than $3 km/s$).


\section{Potential Applications}
\begin{itemize}

\item Railguns are being researched as weapons that would use neither explosives nor propellant.The absence of explosive propellants or warheads to store and handle conventional weaponry come as additional advantages.

\item In addition to military applications, NASA has proposed to use a railgun to launch wedge-shaped aircraft with scramjets to high-altitude at Mach 10, where they will then fire a small payload into orbit using conventional rocket propulsion.

\item Railguns can potentially be used to aid mining, as a substitute for dynamite for clearing tunnels. 

\end{itemize}


\section{Contributions}

There are three types of asynchronous counter-
\begin{itemize}
	\item n-bit Up-Counter
	\item n-bit Down-Counter
	\item n-bit Up/Down-Counter 
\end{itemize}

\section{Approach}
We began by deciding the architecture of our gun.
After brainstorming many different setups, we settled on 

\itemize 
\item A set of 2 parallel rails
\item A sherical projectile which shorts the two rails, and reduce frictional losses.
\item A current carrying coil to apply an external, supporting magnetic field.
\item A capacitor bank in order to deliver high currents in a short amount of time to the rails.
 
	We began by designing the Capacitor bank charging circuit, using simulink (A MATLAB simulation software).

	A detailed schematic of the circuit can be found below in Figure 1.
	\\
	
Figure 2 shows the voltage vs time plot of the charging capacitors.
	
\begin{figure}[htp]
	\caption{Circuit Diagram of simulated charging circuit}
	\includegraphics[width=\linewidth]{Circuit.png}
\end{figure}
\newpage	
	
\begin{figure}[h]
	\caption{Plot of voltage across capcitor bank vs time }
	\includegraphics[width=\linewidth]{charging.jpg}
\end{figure}
		
	
We ran some calculations using a matlab script and plotted the force on the projetile vs time along with the force on the projectile vs time.(Figure 3)
\\ \\
	The script and all other code used in this project can be found in the project github repository.
\href{https://github.com/AravindGanesh/IDP-Sem3}{\textbf{https://github.com/AravindGanesh/IDP-Sem3}}








\section{Arduino Code}
\subsection{Binary Counter}
	\begin{lstlisting}[frame=single]
	int a,b,c,d,e,f,g;
int led=8,temp;
int current;
int initial;
int pin1=0;
int A=0,B=0,C=0,D=0;

void setup()
{
  pinMode(0,OUTPUT);
  pinMode(1,OUTPUT);
  pinMode(2,OUTPUT);
  pinMode(3,OUTPUT);
  pinMode(4,OUTPUT);
  pinMode(5,OUTPUT);
  pinMode(6,OUTPUT);
  pinMode(7,OUTPUT);
  pinMode(8,OUTPUT);
  pinMode(9,INPUT);
  pinMode(10,INPUT);
  pinMode(11,INPUT);
  pinMode(12,INPUT);
  pinMode(13,OUTPUT);
}



void loop()
{
  digitalWrite(led, HIGH);   
  delay(10);
  digitalWrite(led, LOW);  
  
  A=digitalRead(9);
  B=digitalRead(10);
  C=digitalRead(11);
  D=digitalRead(12);
  /*Serial.print(A);
  Serial.print(B);
  Serial.print(C);
  Serial.print(D);
  
  Serial.print("\n");*/
  
  initial=millis();
  current=initial;
  while (current-initial<=2000)
   {
      current=millis();
      digitalWrite(1,1);
      digitalWrite(0,0);
    
      a = (!A && B && !C && !D)
      || (!A && !B && !C && D) 
      || (A && B && C && !D) 
      || (A && !B && C && D);
      b = (!A && B && !C && D) 
      || (!A && B && C && !D) 
      || (A && B && !C && !D) 
      || (A && B && C && D);
      c = (!A && !B && C && !D);
      d = (!B && !C && D) 
      || (!A && B && !C && !D) 
      || (A && !B && D) 
      || (!A && B && C && D) 
      || (A && B && C && !D);
      e = (D)
      ||(!A && B && !C) 
      || (A && B && C);
      f = (!B && C&& D) 
      || (!A && !B && D) 
      || (D && !A && C) 
      || (C &&!A && !B)
      || (!C &&A && B);
      g = (!A && B  && C && D) 
      || (!A && !B && !C) 
      || (A && !B && C);
      
      
if(A==1 && B==1 &&C==0 &&D==0)
      {
        temp=b;
        b=c;
        c=temp;
      }
      
      writenum(a,b,c,d,e,f,g);
      delay(10);
      
      digitalWrite(0,1);
      digitalWrite(1,0); 
      
       if (A==1&&(B==1||C==1))
      {
         writenum(1,0,0,1,1,1,1);
      }
      else
      {
         writenum(0,0,0,0,0,0,1);
      }
      delay(10);
  } 
}

void writenum(int a,int b,int c,
int d,int e,int f,int g)
{
 digitalWrite(13,a);
 digitalWrite(2,b);
 digitalWrite(3,c);
 digitalWrite(4,d);
 digitalWrite(5,e);
 digitalWrite(6,f);
 digitalWrite(7,g);
}


	\end{lstlisting} 
\section{Verilog Code}
\subsection{Behavioural Model}
	\begin{lstlisting}[frame=single]
module counter(clk,reset,Q);

input clk;
output[3:0] Q;
input reset;

wire clk,reset;
reg[3:0] Q;

initial Q=0;

always @ (posedge(clk)) 
	begin
	if (reset)		
		Q<=0;		
	else 		
		Q<=Q+1;
	end	
endmodule
	\end{lstlisting}
	
\subsection{Structural Model}
\begin{lstlisting}[frame=single]
//module
module DFF(d,clk,out); 
	input d;
	input clk;
	output reg out;
	
	always@(posedge clk)
		begin
			out<=d;
		end
		
endmodule

module binarycounter
(clk,out1,out2,out3,out4);
  input        clk;
  output out1; 
  output out2; 
  output out3; 
  output out4; 
  
  DFF dff1(~out1,clk,out1);
  
  DFF dff2(~out2,out1,out2);
  
  DFF dff3(~out3,out2,out3);
  
  DFF dff4(~out4,out3,out4);
endmodule


\end{lstlisting}	
\subsection{Behavioural Testbench}
	\begin{lstlisting}[frame=single]
`timescale 1ns/100ps
`include "counter.v"

module Top;
reg clk;
reg reset;
wire[3:0] Q;
counter c1(clk, reset , Q);
initial 
	begin
	reset = 0;
	#1000 reset=1;	
	#100 reset=0;
	#5000 reset=1;
	#100  reset=0;
	#10000 $finish;
	end
always 
	begin
	clk = 0; #50;
	clk = 1; #50;
	end

initial

begin

	

$monitor($stime,"
 clk = %b,
  reset = %b,
   dout = %h",
   clk,reset,Q );
$dumpfile("counter.vcd");
$dumpvars;

end
endmodule 	
	
	\end{lstlisting}
	

\section{Tests}
To test the circuit, we used the following parameters and conditions - 
\begin{itemize}
	\item Since the ICs and components are DC grade, we are using only 5 volt as input for power and clock signal amplitude.
	\item We passed Clock signals of various frequencies and checked for the output
	\item We tested the circuit with only LEDs for binary output, with a clock pulse from arduino
	\item Tested the circuit with one 7 segment display with same clock pulse
	\item Tested the circuit with two multiplexed displays. The clock pulse had a different duty cycle due to multiplexing logic.  
\end{itemize}

\section{Results}

	Figure 2 shows the RTL synthesis of the 4 bit counter.
The code used can be found in the structural model in section IX.

Figure 3 shows the simulated waveform of the 4 bit counter in GTKwave.
The code used can be found in the Behavioural model and testbench in section IX. 



\section{Future Work}
\begin{itemize}
\item Increase the number of bits of the output using more registers.
\item Utilization of potentiometer to vary the speed of  counting.
\item Implementation of decoder to 7 segment display in hardware, and not using microcontroller.
\item Supplying clock signal without using a Arduino(ie. 555 timer circuit)

\end{itemize}


\section{Bibliography} 
\begin{itemize}
\item https://www.allaboutcircuits.com.

\item electronics-course.com/ripple-counter.

\item https://www.eecs.tufts.edu/~dsculley/tutorial

\end{itemize}
%----------------------------------------------------------------------------------------

\end{document}
